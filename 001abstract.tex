Fecal contamination in bodies of water is an issue that frequently plagues public and environmental water supplies.
%----------%
Often, restricting access to water sources until the contaminants dissipate is the only option natural resource managers have.
%----------%
Finding the source of the fecal matter can help prevent further contamination and more quickly curb the issue in the future.
%----------%
\MSTlong{} (\mst{}) is the field of study that aims to determine the source \spec{} of strains of microbiological lifeforms and \libdep{} \mst{} is one method that can assist in this fecal matter sourcing.
%----------%
Recently, the Biology Department in conjunction with the Computer Science Department at \cplong{} (\cp{}) teamed up to build a database called \cploplong{} (\cplop{}).
%----------%
Students collect fecal samples, culture \isols{} and pyrosequence the two \itslong{} (\itsshort{}) DNA regions of the \ecoli{} in the samples, and insert this data, called pyroprints, into \cplop{}.
%----------%
We investigate two new \mst{} methodologies: one that uses \dbscan{} to cluster for \bslongs{} and another set of algorithms called the \kraplong{} (\krap{}) containing four strategies to resolve the multiple \knnlong{} lists that result from applying \knnlong{} to the two \itsshort{} regions of an \ecoli{} \isol{}.
%----------%
By using \dbscan{} to build strains, we found many pure strains, validating the effectiveness of using \ecoli{} in \cplop{}, and verified the existence of transient strains, but discovered that it left much of the data remained unclustered.
%----------%
As a fallback, we turn to \krap{}, which allows us to perform \mst{} in a straightforward way with high accuracy for \spec{} well-represented by \cplop{}.