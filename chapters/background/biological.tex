\section{Biological}
\subsection{Fecal Contamination}
Fecal contamination is dangerous for humans and animals alike.
Pathogenic bacteria reside in fecal matter and contamination of publicly accessible and environmental water sources expose humans and animals to their dangerous effects.
Often, it is up to natural resource managers and public health officials to both eliminate and prevent fecal contamination from occurring.
Preventing contact with such bacteria is key to maintaining good public and environmental health.



\subsection{\MSTlong{} with \FIBlong{}}
\MSTlong{} (\mst{}) aims to discover the \spec{} of microbial lifeforms, often employing \FIBlong{} (\fib{}) to discover the source of fecal contamination.
Many techniques exist that leverage unique characteristics of the \fib{} present in the fecal matter.
Ultimately, choosing the right \fib{} requires that researchers and investigators understand their resource and budget constraints, and tailor their \mst{} process accordingly.

\subsection{Delineating Bacterial Strains}
When using \fib{} for \mst{}, distinguishing between \bslongs{} (\bs{}) is a dynamic tool that can allow researchers to determine the source of fecal matter from a variety of \spec{}.
Other methods that rely on \mst{} are usually limited in the \spec{} they are effective at sourcing.
Bacterial strain typing offers a flexible and widely applicable method of sourcing fecal matter.
Nevertheless, it has some drawbacks that, with some effort, \mst{} investigators and researchers can overcome.