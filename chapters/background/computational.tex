\section{Computer Algorithms}
Two computer science concepts are core to the clustering and classification techniques investigated in this thesis.
One is a clustering algorithm that builds clusters by categorizing the datapoints into three different types, clustering some and denoting the unclustered as noise.
Another is a classification technique that searches for nearby datapoints in order to classify an unknown datapoint.


%%%%% DBSCAN %%%%%
\subsection{Density-Based Clustering of \Isols{}}\label{sec:background:dbscan}
Density-based clustering algorithms build clusters based on two parameters: the minimum number of neighbors \minneigh{},  a point must have to be a core point of a cluster, and \eps{}, the radius that those neighbors must be within. 
These algorithms define clusters with respect to core points and border points --- points within \eps{} of a core point --- labeling everything else --- the singletons --- as noise.
For this work, we chose to use \dbscan{} as the clustering technique for grouping \isols{} because dense groupings of similar \isols{} fits our intuition of bacterial \isol{} strains.
Closely related ``families'' of \isols{} will appear in the same cluster and we want these clusters to have sufficient purity to aid us in \mst{}.


\dbscan{}\cite{ester1996density} provides the framework for our clustering algorithm. 
It uses a \distmetric{}, a minimum neighbors value \minneigh{}, and an \eps{} range to categorize data points as one of three types:
\index{\dbscan{}}
\index{\distmetric{}}
\index{\minneigh{}}
\index{\eps{}}
a core point, a border point, or noise.
A \textit{core point} is a point that has at least \minneigh{} data points within \eps{} of it. 
\index{core point}
A \textit{border point} is a point that is within \eps{} of a core point, but that does not have \minneigh{} points within \eps{} of it.
\index{border point}
Every other point is \textit{noise}. 
A \textit{cluster} is a group of neighboring core points with their associated border points.
\index{cluster}
According to this definition of a cluster, all clusters must have at least \minneigh{} points in them.
\autoref{fig:density-based-clustering} depicts this process.
\input{figures/density}

Density-based clustering techniques require a distance metric, often times the \euclid{}, between data points in order to cluster.
Performing fast range queries greatly improves the speed of clustering.
If the range query can finish in \bigo{\log{n}} time, then \dbscan{} can run in \bigo{n\log{n}} time.
Organizing the data into a spatial index can optimize these spatial queries.

Spatial indexes structure the data into a search tree, similar to a binary search tree, organizing the points by distance.
When querying for nearby points, the algorithm can traverse this search tree, ignoring certain points along the way.
While this can speed up the range query to a \bigo{\log{n}}, many spatial indexes degenerate into a \bigo{n} operation. In the former case, this makes \dbscan{} run in \bigo{n\log{n}} time.

In \dbscan{}, the \codefn{RangeQuery} function handles range quer\-ies by taking as parameters the data point and a distance and returning all data points within range of the query point.
Mathematically, a \textit{range query} is a function $\mathit{RangeQuery}: D\times \R \rightarrow \{D\}$ that takes a query point $q\in D$ and a real-valued $\eps{}\in\R$ and returns $\{d\in D | \mathit{Dist}(q, d)\} < \eps{}\}$ --- the set of all other points within \eps{} of the query point --- where $\mathit{Dist}$ is some \distmetric{} $\mathit{Dist}: D\times D \rightarrow \R $.
\index{range query}
If the \distmetric{} forms a proper metric space, then data structures like quad trees and octrees can speedup \eps{} range queries for a point.
One can imagine the range query as a hypersphere centered at the query point with a radius of the query range.
In order to make \codefn{RangeQuery} fast, we had to make some optimizations.


%%%%% kNN %%%%%
\subsection{\kNNlong{}}\label{sec:background:knn}
The \kNNlong{} classification algorithm (\kNN{}) is a straightforward algorithm to classify an unclassified object using a library. 
Using a \compfunc{}, it compares the unclassified object to ``nearby'' classified objects.
It uses the concept of a \compfunc{} to formulate an idea of ``closeness,'' asserting that the similarity an unknown object has to a class of objects relates to the class of the unknown objects itself.
\autoref{fig:knn}


To outline the process:
Given an unclassified object \UNKNOWN{}, a library of classified objects \LIB{}, and a \compfunc{}, \COMP{}:
\begin{enumerate}
\item Compare \UNKNOWN{} to each object in \LIB{} using \COMP{}
\item Add the classified object and the result to a list of neighbors, $N$
\item Sort $N$ by most similar
\item Consider only the top $k$ entries in $N$, called the \knnlong{} \label{knn:filter}
\item Classify \UNKNOWN{} as the \textit{most plural} classification in the \knnlong{} list
\end{enumerate}
\autoref{alg:knn} describes this process in pseudocode.
\begin{algorithm}[H]
\caption{\kNNlong{}}\label{alg:knn}

\textit{Input}:
\begin{itemize}
\item $u\in\UNKN{}$: an unknown \isol{}
\item $\LIB{}\subseteq\ISOL{}$: a library of known \isol{}s
\end{itemize}
\textit{Output}:
\begin{itemize}
\item \SPEC{} value, classifying the unknown \isol{} $u$
\end{itemize}
\textit{Requires}:
\begin{itemize}
\item $k$, $\alpha$, and the \compfunc{} \compfunclabel{} are predetermined
\end{itemize}

\begin{algorithmic}[1]
\Procedure{ClassifyKNN$(u,\LIB)$}{}
\State$N \gets\emptyset$    \COMMENT[.6]{Make nearest neighbors list.}
\For{$p\in\LIB$}            \COMMENT[.6]{For each library element:}
    \State $sim\gets$ \Call{$C_i$}{$u,p$}   \COMMENT[.47]{Compare.}
    \State \add{$(sim,p)$}{$N$}          \COMMENT[.47]{Add to neighbors.}
\EndFor
\State \sort{$N$}{$sim$}      \COMMENT[.6]{Sort by most similar.}
\State$s\gets$ \Call{FindMostPluralSpecies}{$\{n_1,\dots,n_k\,|\,n_i\in N\}$}
\State\Return $s$
\EndProcedure
\end{algorithmic}
\end{algorithm}

%%%%%%%%%% MACROS %%%%%%%%%%
\newcommand{\coordsize}{10}
\newcommand{\gridextra}{0.5}
\newcommand{\circleradius}{0.4}
\newcommand{\unknowncolor}{white}
\newcommand{\acolor}{cpgreen}
\newcommand{\bcolor}{cpgold}
\newcommand{\ccolor}{cpgreengold}
\newcommand{\unknownchar}{?}
\newcommand{\achar}{$A$}
\newcommand{\bchar}{$B$}
\newcommand{\cchar}{$C$}
\newcommand{\ac}{\color{white}}
\newcommand{\drawunknownpoint}[1]{
\draw[thick, fill=\unknowncolor] (#1) circle (\circleradius) node[] {\unknownchar}
}
\newcommand{\drawapoint}[1]{
\draw[thick, fill=\acolor] (#1) circle (\circleradius) node[text=white] {\achar}
}
\newcommand{\drawbpoint}[1]{
\draw[thick, fill=\bcolor] (#1) circle (\circleradius) node[] {\bchar}
}
\newcommand{\drawcpoint}[1]{
\draw[thick, fill=\ccolor] (#1) circle (\circleradius) node[] {\cchar}
}

%%%%%%%%%% GRAPH %%%%%%%%%%
\begin{figure}
\centering
\begin{tikzpicture}[scale=1.25]
    \draw[step=1, gray, very thin] (0-\gridextra,0-\gridextra) grid (\coordsize+\gridextra,\coordsize+\gridextra);
    \draw[thick, ->] (0,0) -- (\coordsize+\gridextra,0);
    \draw[thick, ->] (0,0) -- (0,\coordsize+\gridextra);
    \foreach \x in {1,...,10} % edit here for the numbers
    \draw[shift={(\x,0)},color=black] (0pt,0pt) -- (0pt,-3pt) node[below, fill=white] {$\x$};
    \foreach \y in {1,...,10} % edit here for the numbers
    \draw[shift={(0,\y)},color=black] (0pt,0pt) -- (-3pt,0pt) node[left, fill=white] {$\y$};
    
    \drawunknownpoint{5,5};
    \drawbpoint{6,6}; % 1.414 B
    \drawapoint{3,6}; % 2.236 A
    \drawapoint{4,7}; % 2.236 A
    \drawcpoint{3,3}; % 3.000 C
    \drawbpoint{8,5}; % 2.828 B
    \drawbpoint{6,8}; % 3.162 B
    \drawcpoint{3,2}; % 3.606 C
    \drawcpoint{1,3}; % 4.472 C
    \drawcpoint{1,1}; % 5.657 C
    \drawapoint{1,10}; % 6.403 A
\end{tikzpicture}
\caption{Example graph of datapoints in a coordinate space. 
All but one of the datapoints have a class associated with them, \achar{}, \bchar{}, or \cchar{}. 
The class of one point, denoted by `\unknownchar{}', requires determination. 
\Compfuncs{} like \euclid{} or \manhattan{} may be the most appropriate way to compare these datapoints and \autoref{fig:knn} shows how \euclid{} and various \k{} can classify this point.}
\label{fig:knn-graph}
\end{figure}

%%%%%%%%%% TABLES %%%%%%%%%%
\begin{figure}
\centering
\subfloat[For \k{} = 4, the classification for the unknown-class datapoint is \achar{}.]{
\label{fig:knn:4}
\centering
\begin{tabular}{|c|c|c|c|}             \hline
    \k{} & Class   & Position & Similarity    \\ \hline
         & \unknownchar{}& (5,5) & 0.000       \\ \hline\hline
  \rowcolor{\bcolor}  1    & \bchar{}      & (6,6) & 1.414       \\ \hline
  \rowcolor{\acolor}  \ac2    & \ac\achar{}      & \ac(3,6) & \ac2.236       \\ \hline
  \rowcolor{\acolor}  \ac3    & \ac\achar{}      & \ac(4,7) & \ac2.236       \\ \hline
  \rowcolor{\ccolor}  4    & \cchar{}      & (3,3) & 2.828       \\ \hline
%  \rowcolor{\bcolor}  5    & \bchar{}      & (8,5) & 3.000       \\ \hline
%  \rowcolor{\bcolor}  6    & \bchar{}      & (6,8) & 3.162       \\ \hline
%  \rowcolor{\ccolor}  7    & \cchar{}      & (3,2) & 3.606       \\ \hline
%  \rowcolor{\ccolor}  8    & \cchar{}      & (1,3) & 4.472       \\ \hline
%  \rowcolor{\ccolor}  9    & \cchar{}      & (1,1) & 5.657       \\ \hline
%  \rowcolor{\acolor} \ac10    & \ac\achar{}      & \ac(1,10) & \ac6.403       \\ \hline
\end{tabular}
}
\\
\subfloat[For \k{} = 6, the classification for the unknown-class datapoint is \bchar{}.]{
\label{fig:knn:6}
\centering
\begin{tabular}{|c|c|c|c|}             \hline
    \k{} & Class   & Position & Similarity    \\ \hline
         & \unknownchar{}& (5,5) & 0.000       \\ \hline\hline
  \rowcolor{\bcolor}  1    & \bchar{}      & (6,6) & 1.414       \\ \hline
  \rowcolor{\acolor}  \ac2    & \ac\achar{}      & \ac(3,6) & \ac2.236       \\ \hline
  \rowcolor{\acolor}  \ac3    & \ac\achar{}      & \ac(4,7) & \ac2.236       \\ \hline
  \rowcolor{\ccolor}  4    & \cchar{}      & (3,3) & 2.828       \\ \hline
  \rowcolor{\bcolor}  5    & \bchar{}      & (8,5) & 3.000       \\ \hline
  \rowcolor{\bcolor}  6    & \bchar{}      & (6,8) & 3.162       \\ \hline
%  \rowcolor{\ccolor}  7    & \cchar{}      & (3,2) & 3.606       \\ \hline
%  \rowcolor{\ccolor}  8    & \cchar{}      & (1,3) & 4.472       \\ \hline
%  \rowcolor{\ccolor}  9    & \cchar{}      & (1,1) & 5.657       \\ \hline
%  \rowcolor{\acolor} \ac10    & \ac\achar{}      & \ac(1,10) & \ac6.403       \\ \hline
\end{tabular}
}
\\
\subfloat[For \k{} = 9, the classification for the unknown-class datapoint is \cchar{}.]{
\label{fig:knn:9}
\centering
\begin{tabular}{|c|c|c|c|}             \hline
    \k{} & Class   & Position & Similarity    \\ \hline
         & \unknownchar{}& (5,5) & 0.000       \\ \hline\hline
  \rowcolor{\bcolor}  1    & \bchar{}      & (6,6) & 1.414       \\ \hline
  \rowcolor{\acolor}  \ac2    & \ac\achar{}      & \ac(3,6) & \ac2.236       \\ \hline
  \rowcolor{\acolor}  \ac3    & \ac\achar{}      & \ac(4,7) & \ac2.236       \\ \hline
  \rowcolor{\ccolor}  4    & \cchar{}      & (3,3) & 2.828       \\ \hline
  \rowcolor{\bcolor}  5    & \bchar{}      & (8,5) & 3.000       \\ \hline
  \rowcolor{\bcolor}  6    & \bchar{}      & (6,8) & 3.162       \\ \hline
  \rowcolor{\ccolor}  7    & \cchar{}      & (3,2) & 3.606       \\ \hline
  \rowcolor{\ccolor}  8    & \cchar{}      & (1,3) & 4.472       \\ \hline
  \rowcolor{\ccolor}  9    & \cchar{}      & (1,1) & 5.657       \\ \hline
  %\rowcolor{\acolor} \ac10    & \ac\achar{}      & \ac(1,10) & \ac6.403       \\ \hline
\end{tabular}
}
\caption{A simple \kNNlong{} example, showing how the resultant classification of an unknown-class datapoint can change simply by adjusting the value of \k{}. This example uses the data in \autoref{fig:knn-graph} with similarity values calculated by the \euclid{} of the two points.}
\label{fig:knn}
\end{figure}

The motivation is that the unclassified object must be ``close'' to some of the classified objects in our database, using an appropriate measure of closeness --- the \textit{\compfunc{}} --- for the data. 
\index{\compfunc{}}
By choosing the \textit{most plural} or \textit{dominant} classification --- the classification that shows up the highest number of times ---  in the \knnlong{} we can, with some accuracy, classify our unknown object.
\index{most plural classification}
\index{dominant classification}

\autoref{fig:knn} depicts an example graph of datapoints in a coordinate space. 
All but one of the has a class associated with it, any of \achar{}, \bchar{}, or \cchar{}. 
The class of one point, denoted by \unknownchar{}, requires determination. 
\Compfuncs{} like \euclid{} or \manhattan{} may be the most appropriate way to compare these datapoints.

\autoref{fig:knn} shows how using \kNN{} with \euclid{} and various \k{} can classify this point.
Figures~\ref{fig:knn:4}, \ref{fig:knn:6}, and \ref{fig:knn:9} show the \knnlong{} lists as \k{} changes from 4, to 6, to 9.
At \k{}=4, we see that \kNN{} classifies the unknown as \achar{} because there are 2 \achar{}s, but only one each of \bchar{} and \cchar{}.
For \k{}=6, \bchar{} is the resulting classification, since \bchar{} shows up 3 times --- more than the 2 \achar{} and 1 \cchar{}.
Finally, as we extend \k{} all the way out to 9, \kNN{} classifies it as \cchar{}, since such an extension exposes the \knnlong{} list to all 4 \cchar{}'s, more than any other classification.