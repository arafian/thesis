% STRAINS --------------------
\section{Discussion}\label{sec:discussion:clustering}

In general, we observe two trends in our data. For the \isols{} that get clustered into strains,
our approach correctly identifies the \spec{} with over 80-85\% accuracy. This accuracy
is sufficient to conduct sophisticated \mst{} studies. Most of the strains discovered 
in the \cplop{} data show high degree of purity, and even considering the presence of 
a few large impure clusters, most of the clustered \isols{} fall into strains of high purity.


At the same time, the pure strain-based approach suffers from a drop in the coverage as the size of a cluster grows. 
This means that in general \cplop{} \isols{} tend to be very diverse
and come from strains for which not enough DNA material has been collected and pyrosequenced.
Identifying the \spec{} for \isols{} that do not fall into strains/clusters using the pure strain-based
method is impossible. In future work, our goal is to combine the \kNN{}-based \mst{} method
of \cite{DBLP:conf/bibm/McGovernDKBVG15} with the strain-based approach discussed in this paper
to increase coverage while preserving the high \mst{} accuracy.




%%%%%%%


%Considering how many \isols{} end up in clusters of high purity, as \autoref{fig:clust_purity_dist_3} shows, this occurrence appears to be infrequent.
%Nevertheless, viewing the progression of graphs in \autoref{fig:clust_pure} implies that the clusters morph and change into each other as we increase \minneigh{}.

One factor explaining the large impure clusters is the possibility that these clusters represent what the  biologists call  ``transient'' strains, i.e., strains that  persist in more than one \spec{}. 
%That is, certain strains might show up in many \spec{} and not just relegated to one \spec{}.
Such a characteristic can compound \mst{} by making certain strains of \ecoli{} less reliable as \fib{} for identifying \spec{}.
In \autoref{fig:clust_purity_dist}, we see evidence of that and it is revealed in \autoref{fig:clust_pure}.
One mitigation strategy may be to reduce the presence of these strains in the library holding the \fib{}.
Another may be to fall back to an alternative \mst{} technique that works with \cplop{} when an unknown \isol{} falls into an impure cluster.
Finally, if a true transient strain is indeed discovered, and an \isol{} is mapped to it, our \mst{} procedure can simply acknowledge that the query \isol{} belongs to a transient strain and provide information about the \spec{} that show high frequency of \ecoli{} incidence from this strain.
In order to handle the lack of complete clustering coverage --- when \dbscan{} marks an \isol{} as noise --- we propose a fallback method : the \kraplong{}, described in \autoref{chap:krap}.


%It also is possible that a more complicated or biologically-motivated strategy may work best.
%One avenue of validation we chose not to pursue was $k$-fold cross-validation.
%Previous work \cite{DBLP:conf/bibm/McGovernDKBVG15} has used it for validation.
%Future work may include it, but large $k$ values are likely to partition the dataset into groups different enough to create widely varying clusterings.