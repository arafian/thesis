\section{Methodology}\label{sec:methodology:clustering}

In order to understand the make up of \bslongs{} in \cplop{}, we chose to investigate how a density-based clustering algorithm might group the \isols{} we have collected so far and how that might affect a rudimentary \mst{} technique based off of clustering: cluster an unknown \isol{} along with the \isols{} in \cplop{} and classify it as the most plural species of the cluster.
Density-based clustering ties in well with our notion of closely-related strains of \ecoli{} (the relation being the separate \pearson{} comparison of each \itsshort{} region we use to compare \isols{}).
\index{\ecoli{} strain}
From the computer science point of view, a bacterial strain is essentially a cluster of \ecoli{} \isol{} representations stored in \cplop{}.
Our \mst{} method, thus, works as follows:
\begin{enumerate}
    \item \textbf{Strain Identification.} Identify bacterial strains in \cplop{} by clustering
    all \cplop{} \isols{}.
    \item \textbf{MST.} Given an isolate of unknown origin, find the cluster it belongs to.
    Return the \spec{} of the plurality of isolates in the cluster.
\end{enumerate}

Our clustering algorithm is the density-based clustering algorithm developed by Johnson \cite{johnson2015density}.
It extends \dbscan{} for the case of two \compfuncs{} between data points (our isolates are compared based on the two \itsshort{} regions) and implements an efficient spatial data structure to manage the retrieval of the data points.

\dbscan{} can easily use an efficient range query technique to find nearby points and speed up clustering time considerably by taking advantage of the \trieq{} that proper metric spaces have.
Unfortunately, \pearson{} does not encode a metric space, because it fails the \trieq{}\footnote{$d(x,z)\leq d(x,y) + d(y,z)$}. 
This complicates range queries, discussed in \autoref{sec:dbscan}, because spatial indexes tend to rely on the \trieq{}, usually with \euclid{}, to argue that certain points can be ignored during a spatial index tree traversal. 

To allow us the use of a spatial data structure with the \pearson{} to store data points during the clustering procedure we use instead the \clustplop{}, which we derive by recognizing in \autoref{eq:pearson} that \pearson{} is made up of \zscores{} of \pcveca{} and \pcvecb{}. 
The \zscore{} of \pcveca{} is:
\[
\zscoreeq{\pca}
\]
where \vecavg{\pca} and \vecstddev{\pca} are the mean and standard deviation of the values in a single pyroprint respectively. 
Thus, for clustering, we compare \pyros{} using the Euclidean distance \euczfunclabel{} of $z$-scores.
\begin{equation}\label{eq:euclidean_zscores}
\eucz{\pca}{\pcb}
\end{equation}
where \numdims{} is the number of dimensions.
This allows us to use spatial indexes and \bigo{\log{n}} lookup in \dbscan{}.

Each \isol{} is represented in \cplop{} by a pair of \pyros{}: one each from each \Ssixt{} and \Sfive{} region, complicating the use of \dbscan{} and the meaning of $\alpha$ threshold.
We handle this in \dbscan{} by performing two range queries, one each for \Ssixt{} and \Sfive{}, taking the intersection of the two results.
We must, however, pick a suitable \eps{} for each \itsshort{} region.

\cplop{} uses a threshold value of $\alpha = 0.995$  to compare two \pyros{}.
\Pyros{} with \pearson{} above $\alpha$ are considered to represent the same \dna{} material, while \pyros{} with a \pearson{} below $\alpha$ are considered to represent different \dna{} material \cite{Shealy:SeniorProject, soliman2013cplop, SolimanDVMBNWKG12}.

The number of dispensations \numdims{} used to build a \pyro{} differ for the \Ssixt{} and \Sfive{} regions.
Because  $\numdims{}_{\Ssixt{}} \neq \numdims{}_{\Sfive{}}$, the original $\alpha$ under the space defined by \eqref{eq:euclidean_zscores} no longer applies in the same way to both regions.
An alternative formulation of \eqref{eq:euclidean_zscores}, with respect to the \pearson{} \pcfunclabel{}, is:
\begin{equation}\label{eq:euclidean_zscores_alternate}
\euczalternate{\pca}{\pcb}
\end{equation}
where \numdims{} is the number of dimensions and $\numdims{}_{\pcveca{}} = \numdims{}_{\pcvecb{}} = \numdims{}$. 
Using \autoref{eq:euclidean_zscores_alternate}, we can convert $\alpha$ to the values in \autoref{tab:converted_thresholds}. 
We use these converted $\alpha$ values as the \eps{} for each \itsshort{} region's \codefn{RangeQuery}.
\begin{table}
\centering
\caption{Converted $\alpha$ threshold to fit the new metric space defined by \eqref{eq:euclidean_zscores}.}
\label{tab:converted_thresholds}
\begin{tabular}{|c|c|c|}
\hline
\textbf{\itsshort{} Region} & \Ssixt{}     & \Sfive{}     \\
                            & $\alpha$     & $\alpha$     \\ \hline
\pcfunc{\pca}{\pcb}         & 0.995        & 0.995        \\ \hline
\numdims{}                  & \Ssixtdims{} & \Sfivedims{} \\ \hline
\euczfunc{\pca{}}{\pcb{}}   & 0.9747       & 0.9644       \\ \hline
\end{tabular}
\end{table}

When clustering \cplop{} isolates using our density-based clustering algorithm, we need to set up the two parameters at our disposal: \minneigh{} and \eps{}.
For \eps{} we choose the two values shown in Table \ref{tab:converted_thresholds} converted from the 0.995 \pearson{} threshold of pyroprint similarity. 
Essentially, we only want to consider the \eps{}-neighborhood of a \pyro{} that contains the other \pyros{} that we consider to represent the same DNA material.

For the \minneigh{} parameter, we use \textit{grid search} running our clustering with \minneigh{} set to $1, 2, 3, 4, 5, 6$, and $7$.
The \minneigh{} value adjusts how strict our definition of a cluster is.
That is, the higher the value of \minneigh{}, the more neighbors a core point must have with \eps{} of it and its neighbors to become a cluster.
Balancing this value with the coverage of our algorithm is crucial to its success, because for too low of a value, we may not have a clear plurality in a cluster, while too high of a value may miss some smaller clusters that might classify our unknown \isol{} into something other than noise.
