\section{Cluster Purity and Clustering Coverage}\label{sec:eval:cluster}
Clustering for \bslongs{} has two aspects that we must evaluate: how pure the \bslongs{} (clusters) are and how many of the \isols{} in \cplop{} end up in a cluster (as opposed to noise).
The former tests whether the \ecoli{} strains stay relatively unique to the \spec{} from which they come from, the core theory of library-based \mst{}.
The latter tells us how effective 
\autoref{sec:method:clustering} describes how we can use clustering as a \mst{} method --- cluster an unknown \isol{} along with the \isols{} in \cplop{} and classify it as the most plural species of the cluster.
Cluster purity can easily gauge how effective this technique is at \mst{}, since the concept readily summarizes to how pure, from a \spec{} perspective, a cluster is.
\Dbased{} clustering algorithms such as \dbscan{} may not cluster every datapoint, as mentioned in \autoref{sec:dbscan},  labeling some datapoints (\isols{}) as noise and thus leaving them unclustered.
%Thus, it is crucial that our \mst{} method has the ability to assert a classification on any \isol{} we provide it, driving us to use \kNN{} for classification.
%Nevertheless, by applying \dbscan{} to \cplop{} \isols{}, we gained insight into the nature of \ecoli{} strains that we discuss in \autoref{sec:results:clustering}.

In this paper we look at the results of clustering \cplop{} data using this algorithm from the perspective of cluster purity. We call a cluster (\bslong{}) \textit{100\% pure}
if all isolates that belong to it come from the same \spec{}. 

Of interest to us is the following information:
\begin{enumerate}
    \item The number of 100\% pure clusters and the percentage of bacterial isolates from \cplop{} clustered into pure clusters.
    \item The structure of impure clusters: specifically, whether a dominant \spec{} can
    be clearly identified in each cluster.
    \item Coverage: the total number of \cplop{} isolates found to belong to a strain.
    \item MST Accuracy: the percentage of isolates for which the strain-based MST procedure produces the correct response.
\end{enumerate}
Thus, our core measure is \textit{cluster purity}, the proportion of a cluster that comes from the most plural \spec{} of that particular cluster.
\index{cluster purity}
A \textit{100\% pure cluster} is a cluster which only contains data points (\isols{}) with the same class label (same \spec{} of origin). 

Consider a cluster $C=\{c_1,\ldots, c_K\}$. Let $s(c)$ refer to the species of isolate $c$.
Let $m$ be the plurality species label for data points in $C$, and let the total number of points in
$C$ with $s(c) = m$ be $s_m$. Then the \textit{individual cluster purity} $\nu$ of cluster $C$ is:
\[
    \nu(C) = \frac{s_m}{K}
\]

In addition to computing the purity of individual clusters we want to have an understanding of the overall purity on the entire dataset.
Given a \textit{clustering} $\mathcal{C} = \{C_1,\dots,C_n\}$ on a dataset, we define the size $\mathcal{M}$ of the set of clusters: 
\index{clustering}
\begin{equation}\label{eq:num_isols}
\mathcal{M} = \sum_{i = 0}^{n} |C_i|
\end{equation}
The \textit{overall clustering purity} is:
\index{overall clustering purity}
\begin{equation}\label{eq:overall_clustering}
\sum_{i=1}^{n} \frac{|C_i|}{\mathcal{M}}\cdot\nu(C_i)
\end{equation}
One can think of \eqref{eq:overall_clustering} as a form of weighted arithmetic mean of the purities, where the size of the cluster adds more weight to the value.

% COVERAGE --------------------
%\subsection{Clustering Coverage} 
\label{sec:validation:coverage}
Coverage of the dataset is important to an effective \mst{} method.
The density-based clustering method we use has one key disadvantage: a clustering run  with the parameter \minneigh{}, treats all points that do not fit into a cluster of size  of at least \minneigh{} as noise.
This means that as the value of \minneigh{} grows, so will the number of \isols{} that do not cluster into a strain.

Given the parameter \minneigh{} of the clustering algorithm, we collect the following four measures, that collectively represent the breakdown of all data points (\isols{}) in \cplop{}:
\begin{enumerate}
    \item \textit{Noise.} Number/percentage of \isols{} clustered as noise points.
    \item \textit{Misses.} Number/percentage of \isols{} from  minority species
    in impure clusters.
     \item \textit{Hits.} Number/percentage of \isols{} from plurality species in
     impure clusters.
     \item \textit{Pure points.} Number/percentage of \isols{} in 100\% pure clusters.
\end{enumerate}
\index{noise}
\index{misses}
\index{hits}
\index{pure points}