\chapter{Introduction}
%%%% PROBLEM
Fecal contamination in public water sources is an issue that health officials and city and county governments must combat regularly.
%Severe health risks result from contact or ingestion of water that contains fecal matter due to pathogens present and the presence of fecal coliform bacteria can cause the 
Pathogens present in fecal matter pose severe health risks to humans and pets
and
the decomposition of fecal coliform bacteria can upset the balance of aquatic ecosystems by depleting dissolved oxygen to low enough levels that it may kill other species in the water.
Such severe threats to the health of humans, pets, and the local ecosystem motivates public health officials to take action in order to mitigate its consequences.
%Fecal matter entering public water supplies motivates public health officials to take action in order to mitigate its consequences.
%Mitigating the consequences of fecal contamination in public water supplies becomes an important task for public health officials.
%Often times, no one observes the cause of the fecal contamination and restricting access to the water supply is the only action that health officials can take.
Often times, no one observes the cause of the fecal contamination, but rising levels of fecal coliform bacteria indicate that fecal matter contamination has occurred.
In these situations, usually the only course of action that natural resource managers have is to simply restrict public access until contamination levels reach an acceptable level, which does not actively prevent further contamination.
Identifying the source of fecal contamination in water supplies is an important initial step to prevent further contamination.
%It suits public health officials to identify the source of the contamination and take appropriate steps to prevent further contamination.

%%%% BACKGROUND: MST
\MSTlong{} (\mst{}) is the field of research that aims to discover the \spec{} that microbial lifeforms originate from
and
aids the process of sourcing fecal contamination. 
Microbes thrive inside the gut of animals, as well as in masses of plant matter, and routinely make their way into the environment via fecal matter deposition.
Biologists conjecture that strains of microbes or bacteria present in fecal matter, called \fiblong{} (\fib{}), remain relatively unique to the species of the host they originated from.
A strain of a species of microbe is a subtype of that species where the microbes in that strain are closely related in some meaningful way.
How researchers specifically define strains often differs, since each definition of a strain depends on the characterization of the microbe in question and the methods used to derive such characterizations.
Typically, the objective is to discover which microbes of a bacterial isolate came from the same parent microbe \cite{Li892}.
A strain, then, can be thought of as consisting of individuals descending from the same individual to generate a ``group'' or ``family.''
Researchers put significant effort into using the relevant microbes and appropriately characterizing them in order to discover which strains tend to belong to which species.

%%%% BACKGROUND: RELATED MST
A common method of \mst{} involves collecting fecal matter from a known \spec{}, culturing \isols{} from relevant microbes in the fecal matter, and building a digital representation of the collected \isols{} for storage into a database and analysis.
Storing an appropriate digital representation allows researchers to perform rigorous analysis and comparison between \fib{} \isols{} collected from different \hosts{} and \spec{}, as well as \isols{} collected from the same \host{}, but at different times.
%The physical \isols{} collected are referred to as the library, while their digital representations are contained with a database.
The data inserted into such a database may range from collection metadata about the microbiome, to a specific microbe characterization, or to any other useful set of metrics that can appropriately profile an entry \cite{ritter2003assessment}.

In this way, researchers build a ``library'' of known-\spec{} \isols{}.
Using this library, researchers can take an environmental sample with \fib{} from an unknown source, process the microbial \isols{} using the same procedure and the known-\spec{} \isols{} in the library, and compare the strain representation of the environmental sample to those in the library to find any close matches.
Since the researchers know the \spec{} of the \isols{} in the library, they can make a reasonable determination of the source of the \isols{} in the environmental sample.
The methods used to compare \isols{} and make assertions depends entirely upon the \fib{}, their method of collection, and their digital representation.

Library-based \mst{} is usually only effective within the region in which the known\spec{} \isols{} came from, making it difficult to build a ``one size fits all'' library.
Companies exist that can, for a fee, attempt to determine the \spec{} of a provided sample.
While these companies exist nationwide, they are few in number and usually cannot build a representative sample every region for an accurate determination.
Additionally, when investigating an incident of fecal contamination, investigators want to send out multiple samples to build reliable evidence for a determination of the source.
As a result, the cost of outsourcing becomes too prohibitive and determinations too inaccurate for it to be an option.
Thus, there exists a need for a cost-effective and accurate method of \mst{} in order to properly tackle the problem of preventing fecal contamination in water supplies.

%%%% BACKGROUND: CPLOP


%%%% BACKGROUND: RELATED CPLOP
%%%% BACKGROUND: kNN CLASSIFICATION
%%%% METHOD: kNN REGION RESOLUTION
%%%% METHOD: CLUSTERING
%%%% CONCLUSION