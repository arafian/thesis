Since each \ecoli{} \isol{} in \cplop{} has a \pyro{} of two separate \itsshort{} regions\footnote{see \autoref{sec:background:isolates}}, we effectively have two \compfuncs{} between datapoints (\isols{}), complicating our use of \kNN{}.
\kNN{} provides \cplop{} biologists a transparent and intuitive way of understanding the \spec{} classification it asserts, so we find that it will be a usefully insightful .
Applying \kNN{} to \cplop{} \isols{} gives us two lists, one for each \itsshort{} region, that we must make a single \spec{} classification from.
In order to accommodate multiple \compfuncs{}, we need a strategy to resolve multiple \knnlong{} lists.
Rather than create a new similarity metric out of a pair of similarity scores, which may deviate from the inherent nature of the \compfunc{}, we choose to update the \kNN{} method with four different ways of selecting the resultant category label: the \kraplong{} (\krap{}).
%We propose the \krap{}, four strategies to resolve these multiple \knnlong{} lists that multiple \compfuncs{} result in.
%Furthermore, \kNN{} provides \cplop{} biologists a transparent and intuitive way of understanding the \spec{} classification it asserts.
We describe these four methods, the evaluation criteria we judge them by, and their results below.

\section{Methodology}\label{sec:methodology:krap}

In what follows, we generalize our problem. Given \UNKNOWN{} and \KNOWN{}, two library objects (\isols{}), and a collection of \compfuncs{}, \COMP{}$=(\COMP{}_1,\ldots \COMP{}_m)$, with $m > 1$, comparing \UNKNOWN{} to \KNOWN{} gives us a collection of values:  
$\COMP{}(\UNKNOWN{},\KNOWN{}) = (\COMP{}_1(\UNKNOWN{},\KNOWN{}),\ldots,\COMP{}_m(\UNKNOWN{},\KNOWN{}))$.
All four resolution procedures described in this section work with such a generalized representation of \isols{} and \compfuncs{} between them.

Given an unknown \isol{} \UNKNOWN{}, a library of classified\footnote{A ``classified \isol{}'' is an \isol{} for which the \spec{} has been identified in the database.} \isols{} \LIB{}, and a set of \compfuncs{} \COMP{}, we compare \UNKNOWN{} to each object in \LIB{} using each \compfunc{} in \COMP{}. To resolve these \compfuncs{}, we propose four algorithms:

\subsection{Comparing \Isols{}}
Comparing \isols{} to each other is of primary interest to biologists using \cplop{}.
%Of primary interest to the biologists using \cplop{} is comparing \isols{} to each other.
\cplop{} represents each \isol{} by a pair of mutually incomparable \pyros{}: one for each of the two \itsshort{} regions.
As a result, given \isols{} $I_1, I_2$, we can represent each as a pair of pyroprint vectors 
\[I_1 = (\vec{q}_1, \vec{q}_2) \text{ and } I_2 = (\vec{r}_1, \vec{r}_2),
\]
where $\vec{q}_1 \text{ and } \vec{r}_1$ are respectively $I_1 \text{ and } I_2$'s \Ssixt{} \pyro{} and $\vec{q}_2 \text{ and } \vec{r}_2$ are respectively $I_1 \text{ and } I_2$'s \Sfive{} \pyro{} \cite{Black2014121}. Since \pyro{}s from different regions are incomparable, comparing \isol{}s must be done as follows:
\[
\COMP{}(I_1, I_2) = (\pcfunclabel(\vec{q}_1, \vec{r}_1), \pcfunclabel(\vec{q}_2, \vec{r}_2)),
\]
where $\pcfunclabel(\cdot,\cdot)$ is between \pyros{} of the same \itsshort{} region and is the \pearson{}. Thus, when comparing \isol{}s, we effectively have two different similarity metrics, one for each \itsshort{} region:
\[
\COMP{}(I_1, I_2) = (\COMPsixt{}(I_1, I_2), \COMPfive{}(I_1, I_2)).
\]

\begin{algorithm}[H]
\caption{Isolate Comparison Metric}\label{alg:compare}
\textit{Input}:
    \begin{itemize}
    \item $u\in\ISOL{}$: an \isol{}
    \item $v\in\ISOL{}$: an \isol{}
    \end{itemize}
\textit{Output}:
    \begin{itemize}
    \item \R{} value, indicating similarity
    \end{itemize}
\textit{Requires}:
    \begin{itemize}
    \item Each \isol{} has a \pyro{} in the $i^{th}$ ITS region
    \end{itemize}

\begin{algorithmic}[1]
\Procedure{$C_i(u,v)$}{}
\State $\vec{p_u} \leftarrow$ \Call{GetPyroprint$_i$}{$u$}
\State $\vec{p_v} \leftarrow$ \Call{GetPyroprint$_i$}{$v$}
\State \Return \Call{PearsonCorrelation}{$\vec{p_u},\vec{p_v}$}
\EndProcedure
\end{algorithmic}
\end{algorithm}

%%%%%%%%%% FILTER %%%%%%%%%%
\subsection{\a{} Filtering}\index{\a{} filtering}
Our first modification to \kNN{} is an additional condition at step \ref{knn:filter}, after finding the \knnlong{}:
\begin{enumerate}
\setcounter{enumi}{3}
\item Consider only the top $k$ entries in $N$ above threshold $\alpha$
\end{enumerate}
The $\alpha$ threshold allows biologists to filter out neighbors that are among the $k$ closest, but too dissimilar to compare. 
When comparing multiple \pyro{}s of the same region of a single \isol{} for quality control, the \pearson{} between them is strictly above 0.995.
As a result, for many other studies --- not necessarily MST-focused --- \cplop{} researchers use a \pearson{} of 0.990 or above to define a strain of \ecoli{}.
Filtering by some value near this may give more accurate results and provides an intuitive way to relate these lists to other studies.

Using the example from Figures~\ref{fig:knn-graph} and \ref{fig:knn}, \autoref{fig:alpha-filter} shows how when using \kNN{} where \k{} = 9, we can further restrict the \knnlong{} list if we believe \kNN{} should ignore any datapoints outside of \a{}.
For \autoref{fig:alpha-filter}, \euclid{} is the chosen \compfunc{} and using an \a{} threshold to filter means we filter any datapoints from the \knnlong{} list if the similarity value is \textit{above} \a{}.
\cplop{} uses \pyros{} as datapoints and the \pearson{} to compare \pyros{};
the \pearson{} similarity value ranges from -1 to 1, where a \pearson{} close to 1 denotes highly similar datapoints.
Thus, filtering using an \a{} threshold on \pearson{} values requires filtering datapoints that are \textit{below} \a{}.
\autoref{alg:filter} describes this process in pseudocode.

\begin{figure}
\centering
\begin{tabular}{|c|c|c|c|}             \hline
    \k{} & Class   & Position & Similarity    \\ \hline
         & \unknownchar{}& (5,5) & 0.000       \\ \hline\hline
  \rowcolor{\bcolor}  1    & \bchar{}      & (6,6) & 1.414       \\ \hline
  \rowcolor{\acolor}  \ac2    & \ac\achar{}      & \ac(3,6) & \ac2.236       \\ \hline
  \rowcolor{\acolor}  \ac3    & \ac\achar{}      & \ac(4,7) & \ac2.236       \\ \hline
  \rowcolor{\ccolor}  4    & \cchar{}      & (3,3) & 2.828       \\ \hline
  \rowcolor{\bcolor}  5    & \bchar{}      & (8,5) & 3.000       \\ \hline
  \rowcolor{\bcolor}  6    & \bchar{}      & (6,8) & 3.162       \\ \hline
  \rowcolor{\ccolor}  7    & \cchar{}      & (3,2) & 3.606       \\ \hline
  \rowcolor{gray}  8    & \cchar{}      & (1,3) & 4.472       \\ \hline
  \rowcolor{gray}  9    & \cchar{}      & (1,1) & 5.657       \\ \hline
  %\rowcolor{\acolor} \ac10    & \ac\achar{}      & \ac(1,10) & \ac6.403       \\ \hline
\end{tabular}
\caption{For \k{} = 9, filtering the \knnlong{} by \a{} = 4.000 ends up dropping the last two datapoints (marked in gray) off of the list, changing the resultant classification from \cchar{} to \bchar{}. In the case of \euclid{}, using \a{} means we filter any \euclid{} \textit{above} \a{}.}
\label{fig:alpha-filter}
\end{figure}

\begin{algorithm}[H]
\caption{\kNN{} with $\alpha$ Threshold}\label{alg:filter}
\textit{Input}:
    \begin{itemize}
    \item $list$: sorted list
    \end{itemize}
\textit{Output}:
    \begin{itemize}
    \item $nearest$: the top $k$ elements above the threshold $\alpha$
    \end{itemize}
\textit{Requires}:
    \begin{itemize}
    \item $k$ and $\alpha$ are predetermined
    \item each element in $list$ has a similarity field $sim$
    \end{itemize}

\begin{algorithmic}[1]
\Procedure{Filter$_{k, \alpha}$}{$list$}
\State $nearest\gets\emptyset$
\State $i\gets0$
\While{$i < k$ \textbf{and} $list[i].sim < \alpha$}
\State \add{$list[i]$}{$nearest$}
\State $i\gets i + 1$
\EndWhile
\State\Return{$nearest$}
\EndProcedure
\end{algorithmic}
\end{algorithm}

%%%%%%%%%% MEAN %%%%%%%%%%
\subsection{\rmean{}}
\rmean{} combines the result of the two comparisons between two \isols{} in order to build a single \knnlong{} list, from which simple \kNN{} proceeds to classify.
Combining the similarity values from the two \compfuncs{} is simply a mapping from the two similarity values to a single value.
Arithmetic mean, geometric mean, and \euclid{} are examples, but we chose to use \euclid{}.
\autoref{fig:meanwise:example} depicts an example classifying an unknown \isol{} \UNKNOWN{} using the arithmetic mean as the mean function.

\begin{figure}
\centering
\begin{tabular}{c|c|c|c|c|c}
    \k{} & \Isol{} $x$ & \Spec{} & $\pcsixt(u,x)$ & $\pcfive{}(u,x)$ & $\mathit{mean}(\pcsixt{}, \pcfive{})$  \\ \hline\hline
    1 & $a$ & Cat       & 0.994 & 0.991 &  0.993 \\ \hline
    2 & $b$ & Dog       & 0.990 & 0.994 &  0.992 \\ \hline
    3 & $c$ & Dog       & 0.995 & 0.989 &  0.992 \\ \hline
    4 & $d$ & Chicken   & 0.985 & 0.987 &  0.986 \\ \hline
    5 & $e$ & Cat       & 0.980 & 0.990 &  0.985 \\ \hline
    6 & $f$ & Cat       & 0.978 & 0.990 &  0.984 \\ \hline
    7 & $g$ & Cat       & 0.980 & 0.984 &  0.982 \\ \hline
    8 & $h$ & Chicken   & 0.952 & 0.960 &  0.956 \\ \hline
    %9 & $i$ & Chicken   & 0.952 & 0.946 &  0.948 \\ \hline
\end{tabular}
\caption{An example classifying an unknown \isol{} \UNKNOWN{} using \rmean{} with \k{} = 8, where the mean function is the arithmetic average of the two \pcfunclabel{} values.}
\label{fig:meanwise:example}
\end{figure}

Formally: For \UNKNOWN{} and a $\SOMEPYRO\in\LIB{}$, we take the mean of the result of all of the \compfuncs{} and build a single \knnlong{} list from it.
The mean can be any metric mapping $\R\times\R\rightarrow\R$ and in the investigated implementation, we use the \euclid{}, also known as the $L^2$ norm.
A single \knnlong{} list results from this algorithm that we filter by $k$ and $\alpha$ and use to classify the unknown.
\autoref{alg:mean} describes this process in pseudocode.
\begin{algorithm}[H]
\caption{\rmean}\label{alg:mean}

\textit{Input}:
\begin{itemize}
\item $u\in\UNKN{}$: an unknown \isol{}
\item $\LIB{}\subseteq\ISOL{}$: a library of known \isol{}s
\end{itemize}
\textit{Output}:
\begin{itemize}
\item \SPEC{} value, classifying the unknown \isol{} $u$
\end{itemize}
\textit{Requires}:
\begin{itemize}
\item $k$, $\alpha$, and the set of comparison metrics \COMP{} are predetermined
\end{itemize}

\begin{algorithmic}[1]
\Procedure{ClassifyMean$(u,\LIB)$}{}
\State$N \gets\emptyset$    \COMMENT[.6]{Make nearest neighbors list.}
\For{$p\in\LIB$}            \COMMENT[.6]{For each library element:}
\State$\AVG{}\gets\{\emptyset\}$    \COMMENT[.55]{Make empty set for results.}
    \For{$C_i\in \COMP{}$}          \COMMENT[.55]{For each comparison metric:}
    \State $sim\gets$ \Call{$C_i$}{$u,p$}   \COMMENT[.3]{Compare.}
    \State \add{$sim$}{\AVG{}}              \COMMENT[.3]{Track result.}
    \EndFor
    \State $mean\gets $\Call{Mean}{\AVG{}}  \COMMENT[.47]{Mean the results.}
    \State \add{$(mean,p)$}{$N$}          \COMMENT[.47]{Add mean to neighbors.}
\EndFor
\State \sort{$N$}{$sim$}      \COMMENT[.5]{Sort by most similar.}
\State $N\gets$\Call{Filter$_{k,\alpha}$}{$N$} \COMMENT[.5]{Keep the nearest.}
\State$s\gets$ \Call{FindMostPluralSpecies}{$M$}
\State\Return $s$
\EndProcedure
\end{algorithmic}
\end{algorithm}

%%%%%%%%%% WINNER %%%%%%%%%%
\subsection{\rwinner{}}
\rwinner{} performs \kNN{} classification on the two \knnlong{} lists resulting from the two \compfuncs{}, \pcfunclabel{} on \Ssixt{} and \pcfunclabel{} on \Sfive{}, picking the \spec{} with the highest number of instances in its original list.
In this way, the strategy picks the ``winning'' classification from the two lists.
\autoref{fig:winner:example} shows how this works.

\begin{figure}
\centering
\subfloat[A \k{} = 8 nearest neighbors list for the \Ssixt{} region of an unknown \isol{}.]{
\begin{tabular}{c|c}
     \k{} (\Ssixt{}) & \Spec{}  \\ \hline \hline
     1 & Bat \\ \hline
     2 & Bat \\ \hline
     3 & Cat \\ \hline
     4 & Pigeon \\ \hline
     5 & Pigeon \\ \hline
     6 & Human \\ \hline
     7 & Human \\ \hline
     8 & Bat \\ \hline
\end{tabular}
}
\quad
\subfloat[A \k{} = 8 nearest neighbors list for the \Sfive{} region of an unknown \isol{}.]{
\begin{tabular}{c|c}
     \k{} (\Sfive{}) & \Spec{}  \\ \hline \hline
     1 & Pigeon \\ \hline
     2 & Bat \\ \hline
     3 & Cat \\ \hline
     4 & Pigeon \\ \hline
     5 & Bat \\ \hline
     6 & Pigeon \\ \hline
     7 & Cat \\ \hline
     8 & Pigeon \\ \hline
\end{tabular}
}
\caption{In this example of \rwinner{} for \k{} = 8, \kNN{} on the \Ssixt{} region results in a classification of Bat, since there are 3 bats in its \knnlong{} list and \kNN{} on \Sfive{} results in Pigeon, since there are 4 Pigeons in its \knnlong{} list. The classification resulting from these two lists is Pigeon, since Pigeon shows up more in its original list.}
\label{fig:winner:example}
\end{figure}

Formally: For each \compfunc{}, we make a \knnlong{} list and filter by $k$ and $\alpha$ accordingly.
Once we finish building each \compfunc{}'s \knnlong{} list, we find the most plural classification from each list and track the number of times that classification shows up in that list.
Then, we classify $u$ based off the classification that has the highest number in its corresponding list.
\autoref{alg:winner} describes this process in pseudocode.
\begin{algorithm}[H]
\caption{\rwinner}\label{alg:winner}

\textit{Input}:
\begin{itemize}
\item $u\in\UNKN{}$: an unknown \isol{}
\item $\LIB{}\subseteq\ISOL{}$: a library of known \isol{}s
\end{itemize}
\textit{Output}:
\begin{itemize}
\item \SPEC{} value, classifying the unknown \isol{} $u$
\end{itemize}
\textit{Requires}:
\begin{itemize}
\item $k$, $\alpha$, and the set of comparison metrics \COMP{} are predetermined
\end{itemize}

\begin{algorithmic}[1]
\Procedure{ClassifyWinner$(u,\LIB)$}{}
\State $N\gets\emptyset$\COMMENT[.6]{New list to track neighbor lists.}
\For{$C_i\in \COMP{}$}                \COMMENT[.6]{For each comparison metric:}
\State$N_i \gets\emptyset$      \COMMENT[.55]{Make nearest neighbors list.}
    \For{$p\in\LIB$}            \COMMENT[.55]{For each library element:}
    \State $sim\gets$ \Call{$C_i$}{$u,p$}   \COMMENT{Compare.}
    \State \add{$(sim,p)$}{$N_i$}           \COMMENT{Add to neighbors.}
    \EndFor
    \State \sort{$N_i$}{$sim$}                          \COMMENT[.45]{Sort by most similar.}
    \State $N_i\gets$\Call{Filter$_{k,\alpha}$}{$N_i$} \COMMENT[.45]{Keep the nearest.}
    \State \add{$N_i$}{$N$}
\EndFor
\State $S\gets\{\emptyset\}$ \COMMENT[.6]{To track each list's most plural.}
\For{$N_i\in N$}
\State$s\gets$ \Call{FindMostPluralSpecies}{$N_i$}
\State\add{$s$}{$S$}
\EndFor
\State\Return \Call{Max}{$S$}\COMMENT[.6]{The most plural overall.}
\EndProcedure
\end{algorithmic}
\end{algorithm}

%%%%%%%%%% UNION %%%%%%%%%%
\subsection{\runion{}}
\runion{} builds a list that is the union of both \knnlong{} lists, classifying the unknown as the most dominant species in the resulting union
It may be that the most dominant \spec{} in each list is different, but a second or third most dominant \spec{} may be more appropriate since it shows up in both lists.
\autoref{fig:union:example} shows an example of this.

In \autoref{fig:union:example}, Figures~\ref{fig:union:example:ssixt} and \ref{fig:union:example:sfive} show the \knnlong{} lists for \Ssixt{} and \Sfive{} respectively.
The dominant \spec{} in each are Human, with 5 instances, and Turkey, with 4 instances.
Combining these into a union of the two in \autoref{fig:union:example:union}\footnote{The union depicted in \autoref{fig:union:example:union} contains the original \k{} for clarity of entry source. These values can help with tie breaking as well, if breaking ties based on average position, or lowest \k{}.}, we see that the dominant \spec{} is Dog with 6 instances in the union, due to the 3 instances of Dog in each \itsshort{} \knnlong{} list.

\begin{figure}
\centering
\subfloat[A \k{} = 8 nearest neighbors list for the \Ssixt{} region of an unknown \isol{}.]{\label{fig:union:example:ssixt}
\begin{tabular}{c|c}
     \k{} (\Ssixt{}) & \Spec{}  \\ \hline \hline
     1 & Dog \\ \hline
     2 & Human \\ \hline
     3 & Human \\ \hline
     4 & Human \\ \hline
     5 & Dog \\ \hline
     6 & Dog \\ \hline
     7 & Human \\ \hline
     8 & Human \\ \hline
\end{tabular}
}
\quad
\subfloat[A \k{} = 8 nearest neighbors list for the \Sfive{} region of an unknown \isol{}.]{\label{fig:union:example:sfive}
\begin{tabular}{c|c}
     \k{} (\Sfive{}) & \Spec{}  \\ \hline \hline
     1 & Turkey \\ \hline
     2 & Dog \\ \hline
     3 & Dog \\ \hline
     4 & Turkey \\ \hline
     5 & Turkey \\ \hline
     6 & Dog \\ \hline
     7 & Turkey \\ \hline
     8 & Cow \\ \hline
\end{tabular}
}
\\
\subfloat[Resulting union of the \knnlong{} lists from each \itsshort{} region.]{
\begin{tabular}{c|c|c}
    Region      & \k{}  & \Spec{}\\ \hline \hline
    \Ssixt{}    & 1     & Dog \\ \hline
    \Sfive{}    & 1     & Turkey \\ \hline
    \Ssixt{}    & 2     & Human \\ \hline
    \Sfive{}    & 2     & Dog \\ \hline
    \Ssixt{}    & 3     & Human \\ \hline
    \Sfive{}    & 3     & Dog \\ \hline
    \Ssixt{}    & 4     & Human \\ \hline
    \Sfive{}    & 4     & Turkey \\ \hline
    \Ssixt{}    & 5     & Dog \\ \hline
    \Sfive{}    & 5     & Turkey \\ \hline
    \Ssixt{}    & 6     & Dog \\ \hline
    \Sfive{}    & 6     & Dog \\ \hline
    \Ssixt{}    & 7     & Human \\ \hline
    \Sfive{}    & 7     & Turkey \\ \hline
    \Ssixt{}    & 8     & Human \\ \hline
    \Sfive{}    & 8     & Cow \\ \hline
\end{tabular}
}
\quad
\subfloat[\Spec{} counts of all three lists, \Ssixt{}, \Sfive{}, and the union of the two.]{\label{fig:union:example:union}
\begin{tabular}{c|c|c}
List & \Spec{} & Count \\\hline \hline
\Ssixt{} & Human & 5 \\\hline
\Ssixt{} & Dog & 3 \\\hline \hline
\Sfive{} & Turkey & 4 \\\hline
\Sfive{} & Dog & 3 \\\hline
\Sfive{} & Cow & 1 \\\hline \hline
Union & Dog & 6 \\\hline
Union & Human & 5 \\\hline
Union & Turkey & 4 \\\hline
Union & Cow & 2 \\\hline
\end{tabular}
}
\caption{In this example of \runion{} for \k{} = 8, we see that when considering each \itsshort{} \knnlong{} list separately, there are two different dominant \spec{}. 
}
\label{fig:union:example}
\end{figure}

Formally: For each \compfunc{}, we make a \knnlong{} list and filter by $k$ and $\alpha$ accordingly.
After building each \knnlong{} list, we combine the lists into a set, keeping track of the original list position for tie-breaking.
From this set, which we dub the union, we count the classifications present in the union and classify $u$ as the most plural in the union of the lists, compared to the other lists.
\autoref{alg:union} describes this process in pseudocode.
\begin{algorithm}[H]
\caption{\runion}\label{alg:union}

\textit{Input}:
\begin{itemize}
\item $u\in\UNKN{}$: an unknown \isol{}
\item $\LIB{}\subseteq\ISOL{}$: a library of known \isol{}s
\end{itemize}
\textit{Output}:
\begin{itemize}
\item \SPEC{} value, classifying the unknown \isol{} $u$
\end{itemize}
\textit{Requires}:
\begin{itemize}
\item $k$, $\alpha$, and the set of comparison metrics \COMP{} are predetermined
\end{itemize}

\begin{algorithmic}[1]
\Procedure{ClassifySetwise$(u,\LIB)$}{}
\State $\MERG{}\gets\{\emptyset\}$    \COMMENT[.6]{Create new common set.}
\For{$C_i\in \COMP{}$}                \COMMENT[.6]{For each comparison metric:}
\State$N_i \gets\emptyset$      \COMMENT[.55]{Make nearest neighbors list.}
    \For{$p\in\LIB$}            \COMMENT[.55]{For each library element:}
    \State $sim\gets$ \Call{$C_i$}{$u,p$}   \COMMENT{Compare.}
    \State \add{$(sim,p)$}{$N_i$}           \COMMENT{Add to neighbors.}
    \EndFor
    \State \sort{$N_i$}{$sim$}    \COMMENT[.55]{Sort by most similar.}
    \State $N_i\gets$\Call{Filter$_{k,\alpha}$}{$N_i$} \COMMENT[.45]{Keep the nearest.}
    \For{$n\in N_i$}            \COMMENT[.55]{For each nearest neighbor:}
    \State \add{$n$}{\MERG{}}\COMMENT{Add to common set.}
    \EndFor
\EndFor
\State$s\gets$ \Call{FindMostPluralSpecies}{$M$}
\State\Return $s$
\EndProcedure
\end{algorithmic}
\end{algorithm}

%%%%%%%%%% INTERSECTION %%%%%%%%%%
\subsection{\rintersect{}}
\rintersect{} works to build a single set of size \k{}, called the intersection, consisting only of \isols{} that appear in both \itsshort{} \knnlong{} lists.
It does this by initially querying for each \itsshort{} region, expanding the list until the intersection is the required size --- \k{}.
This is the most complicated and restrictive of the resolution strategies, possibly requiring \a{} filtering in order to restrict the search.
Filtering by \a{} means the construction on the intersection stops if expanding the \knnlong{} means going outsidethe \a{} threshold, which may result in an intersection that is smaller than \k{}\footnote{Since the same is true for natural \kNN{}, we find this acceptable.}.

\autoref{fig:intersection:example} shows how this strategy proceeds.
In \autoref{fig:intersection:example:ssixt}, we see that there are many Cows in the \Ssixt{} \knnlong{} list, but few of those \isols{} show up in the \Sfive{} \knnlong{} list in \autoref{fig:intersection:example:sfive}.
We then build the intersection, shown in \autoref{fig:intersection:example:intersection}, but there are not enough common \isols{} to make a \k{}=8 sized intersection, so we must extend the initial lists to 13 in order to find enough common \isols{}.
Once we have found enough common \isols{}, \kNN{} can proceed normally on the intersection.

\begin{figure}
\centering
\subfloat[A \k{} = 8 nearest neighbors list including (fabricated) \isol{} IDs for the \Ssixt{} region of an unknown \isol{}, extended in order to find \isols{} in common with the \Sfive{} list, which are bold.]{\label{fig:intersection:example:ssixt}
\begin{tabular}{c|c|c}
 \k{} (\Ssixt{}) & \Isol{} ID & \Spec{} \\ \hline \hline
 \bf 1      &\bf  5823 &\bf Pig \\ \hline
 \bf 2      &\bf  2833 &\bf Pig \\ \hline
 3          &     8873 & Cow \\ \hline
 \bf 4      &\bf  5939 & \bf Cow \\ \hline
 5          &     6156 & Cow \\ \hline
 6          &     3676 & Human \\ \hline
 7          &     7853 & Cow \\ \hline
 \bf 8      &\bf  5331 &\bf Cow \\ \hline\hline
 9          &     2189 & Cow \\ \hline
 \bf 10     &\bf  2053 &\bf Pig \\ \hline
 \bf 11     &\bf  8962 &\bf Pig \\ \hline
 \bf 12     &\bf  3813 &\bf Human \\ \hline
 \bf 13     &\bf  8173 &\bf Pig \\ \hline
\end{tabular}
}
\quad
\subfloat[A \k{} = 8 nearest neighbors list including (fabricated) \isol{} IDs for the \Sfive{} region of an unknown \isol{}, extended in order to find \isols{} in common with the \Ssixt{} list, which are bold.]{\label{fig:intersection:example:sfive}
\begin{tabular}{c|c|c}
 \k{} (\Ssixt{}) & \Isol{} ID & \Spec{} \\ \hline \hline
 \bf 1      &\bf   2833 &\bf Pig \\ \hline
 2          &      2916 & Cow \\ \hline
 \bf 3      &\bf   3813 &\bf Human \\ \hline
 \bf 4      &\bf   5939 &\bf Cow \\ \hline
 5          &      6854 & Dog \\ \hline
 \bf 6      &\bf   5823 &\bf Pig \\ \hline
 \bf 7      &\bf   2053 &\bf Pig \\ \hline
 8          &      8485 & Dog \\ \hline \hline
 \bf 9      &\bf   8173 &\bf Pig \\ \hline
 10         &      6497 & Human \\ \hline
 11         &      9208 & Cow \\ \hline
 \bf 12     &\bf   5331 &\bf Cow \\ \hline
 \bf 13     &\bf   8962 &\bf Pig \\ \hline
\end{tabular}
}
\\
\subfloat[Common isolates in the \Ssixt{} and \Sfive{} \knnlong{} lists, sorted by average \k{}.]{\label{fig:intersection:example:intersection}
\begin{tabular}{c|c|c|c}
\k{} (\Ssixt{}) & \k{} (\Sfive{}) & \Isol{} ID & \Spec{} \\ \hline \hline
 2  & 1 & 2833 & Pig \\ \hline %1.5
 1  & 6 & 5823 & Pig \\ \hline %3.5
 4 & 4 & 5939 & Cow \\ \hline %4
 12 & 3 & 3813 & Human \\ \hline %7.5
 10 & 7 & 2053 & Pig \\ \hline %8.5
 8 & 12 & 5331 & Cow \\ \hline %10
 13 & 9 & 8173 & Pig \\ \hline %11
 11 & 13 & 8962 & Pig \\ \hline %12
\end{tabular}
}
\caption{An example \rintersect{} for \k{} = 8 showing how the original \knnlong{} lists did not have enough \isols{} --- 8 are needed --- in common, but when extended out to 15, there are 8 common \isols{}. From here, classification can proceed to classify the unknown as the most dominant \spec{} in the intersection, Pig.}
\label{fig:intersection:example}
\end{figure}

For each \compfunc{}, we make a \knnlong{} list and filter by $k$ and $\alpha$ accordingly, but ensure that we do not lose track of the entire sorted list of results.
After building each \knnlong{} list, we inspect each list for common \isol{}s.
We add \isol{}s that appear in every list into a set that we call the intersection.
If the size of the intersection is $k$, then we are done.
Otherwise, we increase the length of our individual lists by $\delta$ and search for common \isol{}.
This process repeats until the size of the intersection is $k$, or all of the \isol{}s in the individual lists are below threshold $\alpha$.
\autoref{alg:intersection} describes this function in pseudocode.
\begin{algorithm}[H]
\caption{\rintersect{}}\label{alg:intersection}

\textit{Input}:
\begin{itemize}
\item $u\in\UNKN{}$: an unknown \isol{}
\item $\LIB{}\subseteq\ISOL{}$: a library of known \isol{}s
\end{itemize}
\textit{Output}:
\begin{itemize}
\item \SPEC{} value, classifying the unknown \isol{} $u$
\end{itemize}
\textit{Requires}:
\begin{itemize}
\item $k$, $\alpha$, $\delta$, and the set of comparison metrics \COMP{} are predetermined
\end{itemize}

\begin{algorithmic}[1]
\Procedure{ClassifyIntersection$(u,\LIB)$}{}
\State $N\gets\emptyset$\COMMENT[.6]{New list to track neighbor lists.}
\For{$C_i\in \COMP{}$}                \COMMENT[.6]{For each comparison metric:}
\State$N_i \gets\emptyset$      \COMMENT[.55]{Make nearest neighbors list.}
    \For{$p\in\LIB$}            \COMMENT[.55]{For each library element:}
    \State $sim\gets$ \Call{$C_i$}{$u,p$}   \COMMENT{Compare.}
    \State \add{$(sim,p)$}{$N_i$}           \COMMENT{Add to neighbors.}
    \EndFor
    \State \sort{$N_i$}{$sim$}    \COMMENT[.45]{Sort by most similar.}
    
    \State \add{$N_i$}{$N$}
\EndFor
\State $done\gets$ false
\While{$\neg done$}
\For{$N_i\in N$}
\State $N_i'\gets\emptyset$
\State $N_i'\gets$\Call{Filter$_{k,\alpha}$}{$N_i$}
\EndFor

\If{$|N_1'\cap\dots\cap N_n'| < k$}
\State $k\gets k + \delta$
\Else
\State $N_\cap\gets N_1'\cap\dots\cap N_n'$ 
\State $done\gets$ true
\EndIf
\EndWhile
\State\Return \Call{FindMostPluralSpecies}{$N_\cap$}
\EndProcedure
\end{algorithmic}
\end{algorithm}
