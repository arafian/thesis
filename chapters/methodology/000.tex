\chapter{\MSTlong{} Methodology}\label{chap:methodology}
An effective, automated \mstlong{} (\mst{}) methodology is key to \cplop{}'s success as a tool to aid fecal contamination investigators.
Until recently, \cp{} researchers performed most \mst{} by hand \cite{moritz2015application, shapiro2015source}.
The goal for any \mst{} method is to take fecal matter or a substance contaminated with fecal matter and determine, or classify, the \spec{} that provided the fecal matter; \libdep{} \mst{} methods leverage the known-\spec{} information stored in their library, usually digital representations of \fiblong{} (\fib{}) stored in a database.
\cplop{} is such a \libdep{} technique that 
Given a \fib{} \isol{} from an unknown-\spec{}\footnote{referred to simply as the ``unknown'' or ``unknown \isol{}''}, a \libdep{} \mst{} technique determines the \spec{} of the unknown using the information in the library.
Towards this end, we built and investigated two \mst{} techniques, one that approaches \spec{} classification from the perspective of strains in the database and another that directly uses \isols{} present in the database to classify the \spec{}.
This Chapter outlines the abstract approaches we chose to take, while \autoref{chap:clustering} and \autoref{chap:krap} detail the specific algorithms we used for the strain-based and \isol{}-based approaches.

\section{Strain-Based}
Strain typing is central to \libdep{} \mst{} methods and building a \spec{} classification technique that uses strains directly is an intuitive approach to take.
If an unknown-\spec{} matches a strain in the library, then we can make a reasonable assertion as to its \spec{} if the strain has a dominant \spec{}.
Thus, our approach is: 
\begin{enumerate}
    \item Incorporate the unknown \fib{} \isol{} into the database
    \item Build strains of the \fib{} \isols{}
    \item Classify the source \spec{} of the \isol{} as the dominant \spec{} of the strain it ended up in
\end{enumerate}
Strain construction can happen in many ways, but from a computational perspective, it is very amenable to clustering.
If one can imagine a coordinate space that encapsulates mathematical representations (vectors) of \fib{} \isols{}, then strains are the close groupings (clusters) of these \isol{} representations.
For \cplop{}
As explained in \autoref{chap:clustering}, we use \dbscan{}, a \dbased{} clustering algorithm 

\section{\Isol{}-Based}