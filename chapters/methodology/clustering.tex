\section{Clustering for Bacterial Strains}\label{sec:clusteringbs}
For the purposes of this work, we define an \ecoli{} strain as a \textit{group of \ecoli{} isolates that share exactly the same \Ssixt{} and \Sfive{} DNA sequences.}

From the computer science point of view, a bacterial strain is essentially a cluster of \ecoli{} \isol{} representations stored in \cplop{}.
Our \mst{} method, thus, works as follows:
\begin{enumerate}
    \item \textbf{Strain Identification.} Identify bacterial strains in \cplop{} by clustering
    all \cplop{} \isols{}.
    \item \textbf{MST.} Given an isolate of unknown origin, find the cluster it belongs to.
    Return the \spec{} of the plurality of isolates in the cluster.
\end{enumerate}

For our clustering algorithm, we use a density-based clustering algorithm
developed by Johnson \cite{johnson2015density}. This algorithm extends DBSCAN
for the case of two similarity metrics between data points (our isolates are compared
based on two ITS regions) and implements an efficient spatial data structure to manage
the storage and retrieval of the data points.

In this paper we look at the results of clustering \cplop{} data using this algorithm from the perspective of \textit{cluster purity}. We call a cluster (strain) \textit{100\% pure}
if all isolates that belong to it come from the same \spec{}. 

Of interest to us is the following information:
\begin{enumerate}
    \item The number of 100\% pure clusters and the percentage of bacterial isolates from \cplop{} clustered into pure clusters.
    \item The structure of impure clusters: specifically, whether a dominant \spec{} can
    be clearly identified in each cluster.
    \item Coverage: the total number of \cplop{} isolates found to belong to a strain.
    \item MST Accuracy: the percentage of isolates for which the strain-based MST procedure produces the correct response.
\end{enumerate}

In the next section we provide a brief discussion of the density-based clustering algorithm
of Johnson \cite{johnson2015density} and its use to build \cplop{} \isol{} clusters.
