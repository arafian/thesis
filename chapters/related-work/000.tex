\chapter{Related Work}\label{chap:related-work}

Existing \MSTlong{} (\mst{}) methodologies require a \fiblong{} (\fib{}) fingerprinting method that allows for strain discrimination and a method of classification.
Early work in \mst{} \cite{bitton2005microbial} worked by measuring the ratio of fecal coliform bacteria to streptococci ratios, which fell out of use due to the ``widely varying survival rates of the bacterial groups in the environment'' \cite{sargeant2011review}.
In order to effectively use \fib{} for \mst{}, researchers had to develop new methods to fingerprint and classify them with the appropriate \spec{} or find related strains.

Fingerprinting \fib{} usually falls into two categories: phenotyping and genotyping.
Phenotypic methods of fingerprinting usually involve ``morphology of colonies on various culture media, biochemical tests, serology, killer toxin susceptibility, pathogenicity, and antibiotic susceptibility,'' none of which allows researchers to reliably distinguish between closely related strains \cite{Li892}.
Genetic fingerprinting --- genotyping --- ``has become widely used \dots{} due to its high resolution'' \cite{Li892} and many methods exist that allow for effective discrimination \cite{scott2002microbial, sargeant2011review}.

Classification methods use a variety of statistical measures to make determinations to either related strains or \spec{}, but most fall into library-dependent and library-independent.
Library-independent \mst{} searches for the presence of certain microbes in fecal matter or contaminated water.
The presence of certain microbes can indicate what \spec{} may have deposited the fecal matter.
Unfortunately, this method relies on prior knowledge of the types of microbes that may occur in the types of potential \spec{}\footnote{Often times, these methods can only detect whether the fecal content came from a human and maybe some domestic animal species \cite{sargeant2011review}.}, limiting the effectiveness of \spec{} determination \cite{sargeant2011review}.

Library dependent techniques work by building a database of \fib{} fingerprints that come from known \spec{}.
These techniques usually differ in the fingerprinting process, which the classification technique is dependent upon.
Using these libraries, researchers can handle common \fib{} from a variety of \spec{}, making it incredibly agile.
Disadvantages of this technique come from: the need to build a large library size which can become cost-prohibitive; the transient nature of some \ecoli{} strains, assuming \ecoli{} is the \fib{} of choice; and the fact that the applicability of the database is limited to the region in which the database was built from \cite{sargeant2011review}.
\cplop{} is a library-based \mst{} technique \ecoli{} as \fib{} and \pyros{} as cost-effective fingerprints and researchers use it to understand \ecoli{} strains and determine the \spec{} of fecal matter.

% Pyroprints
\section{\Pyro{}ing}
A key component of strain-based \MSTlong{} is representing strains of \fiblong{} (\fib{}).
Numerous genotypic methods exist for differentiation between strains of \ecoli{}. 
One can find a detailed discussion of why \pyros{} perform better than these options of in \cite{kent2014pyroprinting}.

\cp{} researchers introduce the concept and construction of \pyros{} in \cite{Black2014121, kent2014pyroprinting}, describing the process through which they construct \pyros{} from the multiple loci of isolated \ecoli{} \dna{}.
It discusses the work done in \cite{Shealy:SeniorProject}, which confirmed the reproducibility of \pyro{}ing and determined that a \pearson{} correlation above 0.99 ``could be a good threshold to minimize false separation of isolates from the same strain.''
These works explain in detail the advantages of using the \pyro{}ing methodology with respect to cost (``[p]yroprinting could reduce the cost of a library-based \mst{} investigation by up to 60\%'' \cite{Black2014121}), reproducibility (much of which can be found in \cite{Shealy:SeniorProject}), and discrimination between (known) strains of \ecoli{}, compared to existing state of the art methods.
It asserts that while \ecoli{} were used, the \pyro{}ing process applies to a broad range of bacteria whose genome contain multiple loci.

\Silico{} simulations done in \cite{DBLP:conf/bibm/BrandtMSBGK12} delved into the sensitivity of using the \pearson{} to compare constructed \pyros{} of known \ecoli{} alleles gathered from the \ncbilong{} database.
\cuda{} programming on a GPU sped up the \pearson{} computation considerably, allowing the researchers to understand, given all possible combinations of seven known alleles to form a simulated \isol{}, how many \isols{} are ``\textit{hard to differentiate},'' i.e. have a \pearson{} above 0.99 \cite{DBLP:conf/bibm/BrandtMSBGK12}.
The work in \cite{DBLP:conf/bibm/BrandtMSBGK12} supplements \vitro{} work performed in \cite{DBLP:conf/bibm/MontanaDNBK11}.

% CPLOP
Senior projects and master's theses \cite{soliman2013cplop}, \cite{ricketts2014cal}, and \cite{webb2011cplop} discuss the development of many of the tools in \cplop{}, from the backend database construction, to the frontend web view and usage for investigation.
\cp{} researchers have placed a large emphasis on validation of the methodologies included in building \pyros{} from \ecoli{}\isols{}.
Biology students investigated how \ecoli{} strains change in response to a variety of factors.
Computer science students at \cp{} have developed many tools to aid the biologists in both validating their methodologies and performing \ecoli{} strain research on various \hosts{} and \spec{}

% Real-Life Strain Research
\section{Empirical Strain Research}
\cplop{} has enabled numerous research projects in the field of biology.
The following is a list of empirical strain research performed using \cplop{}:
\begin{itemize}
    \item Demographics and Transfer of \ecoli{} Within Bos taurus Populations \cite{dillard2015demographics}
    \item \ecoli{} Strain Demographics and Transmission in Cattle \cite{dillard2013coli}
    \item Application of Pyroprinting for Source Tracking of E. coli in Pennington Creek \cite{moritz2015application}
    \item Demographics of \ecoli{} Strains in the Human Gut Using Pyroprints:  A Novel MST Method \cite{neal2012demographics}
    \item \ecolilong{} Strain Diversity in Humans: Effects of Sampling Effort and Methodology \cite{neal2013escherichia}
    \item Investigating the Dominant \ecolilong{} Strain in Lambs and Ewes Using Pyroprinting:  A Novel Method for Strain Identification \cite{nguyeninvestigating}
    \item Source Tracking of Fecal Contamination Along San Luis Obispo (SLO) Creek \cite{shapiro2015source}
    \item Short Communication:  Typing and Tracking Bacillaceae in Raw Milk and Milk Powder Using Pyroprinting \cite{vanderkelen2016short}
    \item  \cite{adams2016using}
\end{itemize}

These studies provide significant insight into the evolution and tranmission of \ecoli{} and confirm the effectiveness of using \pyros{} and \cplop{} as a \mst{} method.
Many of the above studies provided a culminating experience for undergraduates and graduates in biology and computer science.
What we aim to provide with \krap{} is a set of tools that students and researchers have at their disposal to make it easier to make discoveries and assertions about strains in \cplop{}.

% Ontological Hierarchical Clustering
\section{Clustering}
Presented in \cite{DBLP:conf/bibm/MontanaDNBK11, montana2012investigating} are the comparison of two hierarchical clustering techniques. \primerfive{} \cite{clarke1993non} and a chronology-sensitive hierarchical clustering algorithm.
Using metadata about when researchers collected the samples used to build the \isols{}, the hierarchical clustering proceeds to first cluster \isols{} from samples collected on the same day and continues to cluster by increasing days away from the initial collection date.
They found that the clusters built by the chronology-sensitive hierarchical clustering algorithm resembled the \primerfive{} clusters, but were unsure of whether these clusters were appropriate.

The work in \cite{DBLP:conf/bibm/MontanaDNBK11, montana2012investigating} went on to become a part of \ohclust{} (\ohclustlong{}) \cite{SolimanDVMBNWKG12, montana2013ontological, montana2013algorithms}, a metadata-aware hierarchical clustering algorithm that allows \cplop{} researchers to provide a metadata ontology to guide the order of hierarchical clustering.
Hierarchical clustering in general is a very calculation-intensive process, making it a problematic tool for servers with limited computational power.
The computational crux comes with the number of comparisons needed between clusters --- clusters of \isols{} in \cplop{}'s case.

Most hierarchical clustering algorithms computer the distances between clusters and agglomerate by picking some number of the clusters to combine into a cluster (of clusters) for the next hierarchy.
Cluster distances are merely the distance between some representative member --- possibly an average of the actual members --- of one cluster with a representative from the other cluster.
The representatives used to compute distance may, for example, be the members in each cluster closest to each other, farthest from each other, or the centroid of each cluster.

Computationally intense distance metrics make implementing a performant hierarchical clustering algorithm problematic for programmers.
The way \ohclust{} gets around this difficulty is by precomputing the distances --- \pearson{}s in \cplop{} --- beforehand and storing them in memory.
This greatly speeds up the clustering, but requires at least 4GB of memory for the distance lookup table alone.
Since the servers that host \cplop{} only have 4GB of RAM in total, \ohclust{} cannot be directly incorporated into \cplop{}.

% Density Based Clustering
In \cite{johnson2015density}, Eric Johnson presents the density-based clustering algorithm for \pyros{} in order to build an intuitive notion of clustering that uses density and nearness, as well as an efficient range query algorithm to find nearby \isols{}.
\dbscan{} \cite{ester1996density}, short for \dbscanlong{}, can be efficient if the distance metric used satisfies the \trieq{}.
Unfortunately, \pearson{} does not satisfy the \trieq{}, but work in \cite{johnson2015density} adjusts the comparison metric to use the \euclid{} of \zscores{} and optimizes further by organizing the \pyros{} into a tree, making \dbscan{} a viable method of clustering for the servers that host \cplop{}.

An attempt to use \cite{johnson2015density} as a na{\"i}ve \mst{} method in \cite{DBLP:conf/bcb/McGovernJDBKV16} revealed that for the \isols{} that actually clustered (i.e. were not determined to be noise), the accuracy was fairly high.
Essentially, \cite{DBLP:conf/bcb/McGovernJDBKV16} clusters an unknown-\spec{} \isol{} along with the rest of the known-\spec{} \isols{} \cplop{} and classifies it as the most plural \spec{} in the resulting cluster.
However, \cite{johnson2015density} clustered only about half of the \isols{} in \cplop{}, while the rest remained unclustered and thus unclassified.
The investigation in \cite{DBLP:conf/bcb/McGovernJDBKV16} was useful to confirm suspicions of so-called ``transient'' strains of \ecoli{} bacteria.
\autoref{sec:dbscan} discusses the details of \cite{johnson2015density} relevant to this thesis and \autoref{sec:clusteringbs} describes a potential methodology to use it as a \mst{} method, while \autoref{sec:clusteringbsresults} expands upon the investigation in \cite{DBLP:conf/bcb/McGovernJDBKV16}.

The clustering methods presented in \cite{DBLP:conf/bibm/MontanaDNBK11, montana2012investigating, SolimanDVMBNWKG12, montana2013ontological, montana2013algorithms} and \cite{johnson2015density} are examples of typical investigations into bacterial strain research.
On their own, they do not constitute an actual \mst{} methodology\footnote{the work in \cite{DBLP:conf/bcb/McGovernJDBKV16} attempts to}.
Ultimately, the goal of \cplop{} is to be able to objectively classify the \spec{} of an \ecoli{} \isol{}.
Thus, merely clustering \isols{} is insufficient for \mst{}.
This thesis presents \krap{}, an \mst{} technique that works with the \pyros{} of \isols{} in \cplop{} as a solution to \mst{}.

\section{\kNN{} Techniques}
A plethora of \kNNlong{} (\kNN{}) methods exist, but most are various attempts to optimize the search space --- either with efficient range queries or by leveraging information about the space to improve search speed --- or modifications to the neighbor list structure to improve classification.
Surveys on \kNN{} techniques \cite{DBLP:journals/corr/abs-1007-0085, DBLP:conf/fskd/JiangCWJ07} show that each variation build data structures for efficient query, or abstracts the notion of the usually \euclid{} metric to build a more accurate classifier, while others may weight the neighbors or remove neighbors from consideration based off of some criteria
An exception to the typically euclidean distance metric comes in the way of recommender systems \cite{DBLP:reference/rsh/DesrosiersK11, DBLP:reference/sp/NingDK15}, which use a notion of similarity based off of scores.
While efficient range query interests us, we have a solution for it in \cite{johnson2015density} 

The method that most closely resembles what we are after comes from techniques that build multiple \kNN{} classifiers by generating feature subsets and polling the classifier to determine the class of the unknown datapoint \cite{DBLP:conf/icml/Bay98, DBLP:journals/ida/Bay99, DBLP:conf/icmlc/WangHWC05}.
Similar to bagging and bootstrapping techniques used train other classification algorithms, the feature-set of known datapoints is either reasonably partitioned into feature-subsets \cite{DBLP:conf/icml/Bay98, DBLP:journals/ida/Bay99}, or clustered into subsets \cite{DBLP:conf/icmlc/WangHWC05}.
The resulting classification from the \kNN{} subset is then aggregated and the final classification is determined by majority voting.
This approach will not apply to \cplop{}, since we do not merely have a single comparison metric that we are partitioning into multiple to improve classification, we always have two entirely separate metrics that we must make a reasonable decision from.

The primary goal of \krapmed{} (\krap{}) is to resolve the two comparison metrics that \cplop{} has for comparing two \isols{}.
That is, given an \isol{}, in order to find nearby \isols{}, one must separately compute the \pearson{} \pcfunclabel{} for each \itsshort{}, giving us two comparison metrics, $\pcfunclabel{}_{\Ssixt{}}$ and $\pcfunclabel{}_{\Sfive{}}$.
Typically, the vectors in \kNN{} techniques represent the entire set of features for a particular datapoint.
Many \kNN{} algorithms assume that the distance metric used --- usually \euclid{} --- will encode a useful notion of distance.

Certain types of data are more complex than what \euclid{} can encode.
For example, in a demographic study for, say, a political study, subjects may have a multitude of features with different metrics of comparison.
Location of residence may be one and favorite color another\footnote{Certainly, many other psychological metrics can exist, but for simplicity's sake, let us consider only these two.}, with a goal of classifying a subject's political party.
Euclidean distance may not be appropriate for the location metric, since great-circle distance on a globe may encode closeness more accurately.
For color, while it may be straightforward to represent red, green, and blue values as a vector, \euclid{} may not be the best choice to gauge similarity in color, certainly not in the same way as the great-circle distance --- \cite{mcleod2014proof} discusses the tremendous difficulties in representing the perceptive color space.
Nevertheless, these metrics on their own are perfectly amenable to their own, accordant distance metric that cannot be used between features, making it easy to create a \kNN{} for each feature separately.
As such, when using \kNN{} on datasets with a complex set of features there is a need for the ability to resolve separate \kNN{} lists in order to usefully classify datapoints.