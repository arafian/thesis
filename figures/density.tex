% MACROS
\newcommand{\dbscanscale}{1.75}
% COLORS
\newcommand{\corecolor}{cpgreen}
\newcommand{\bordercolor}{cpgold}
\newcommand{\noisecolor}{gray}

% CORE
\newcommand{\corepointraw}[1]{
    \draw [fill=\corecolor] (#1) circle (.125)
}
\newcommand{\coreneighborhood}[1]{
    \draw (#1) circle (1)
}
\newcommand{\corepoint}[1]{
    \corepointraw{#1};
    \coreneighborhood{#1};
}

% BORDER
\newcommand{\borderpointraw}[1]{
    \draw [fill=\bordercolor] (#1) circle (.125)
}
\newcommand{\borderneighborhood}[1]{
    \draw[dashed] (#1) circle (1)
}
\newcommand{\borderpoint}[1]{
    \borderpointraw{#1};
    \borderneighborhood{#1};
}
% NOISE
\newcommand{\noisepointraw}[1]{
    \draw [fill=\noisecolor] (#1) circle (.125)
}
\newcommand{\noisepoint}[1]{
    \noisepointraw{#1};
    %\borderneighborhood{#1};
}
\newcommand{\coreradius}{0.80}
\newcommand{\borderradius}{1.4}
\newcommand{\noiseradiussmall}{2}
\newcommand{\noiseradiusmed}{2.85}

% FIGURE
\begin{figure}
\captionsetup[subfigure]{width=\textwidth}
\centering
%\subfloat[
%    In order to cluster using \dbscan{}, one must programmatically inspect the \eps{} neighborhood of each datapoint.
%]{
%    \begin{tikzpicture}[scale=\dbscanscale]
%        \noisepointraw{0:0};
%        \borderneighborhood{0:0};
%    \end{tikzpicture}
%}\quad
%\subfloat[
%    Categorizing datapoints as either Core Points, Border Points, or Noise is simply a matter of counting how many neighbors fall within a point's neighborhood.
%]{
    \begin{tikzpicture}[scale=\dbscanscale]
    % Core Points
    \corepoint{0:0}
    \corepoint{0:\coreradius}
    \corepoint{120:\coreradius}
    \corepoint{240:\coreradius}
    
    % Border Points
    \borderpoint{ 90:\borderradius}
    \borderpoint{150:\borderradius}
    
    \borderpoint{210:\borderradius}
    \borderpoint{270:\borderradius}
    
    \borderpoint{ 30:\borderradius}
    \borderpoint{330:\borderradius}
    
    % Noise Points
    \noisepoint{ 20:\noiseradiussmall}
    \noisepoint{ 90:\noiseradiussmall}
    
    \noisepoint{  0:\noiseradiusmed}
    \noisepoint{180:\noiseradiusmed}
    
    \noisepoint{ 60:\noiseradiusmed}
    \noisepoint{120:\noiseradiusmed}
    \noisepoint{300:\noiseradiusmed}
    \noisepoint{240:\noiseradiusmed}
    
    \noisepoint{220:\noiseradiussmall}
    \noisepoint{260:\noiseradiussmall}
    \noisepoint{340:\noiseradiussmall}
    \end{tikzpicture}
    
    \vspace{12pt}
    \begin{tikzpicture}[scale = \dbscanscale]
    \corepointraw{0:0}       node[anchor=west, xshift=4pt] {Core Point};
    \borderpointraw{270:.4}  node[anchor=west, xshift=4pt] {Border Point};
    \noisepointraw{270:.8}  node[anchor=west, xshift=4pt] {Noise};
    \draw[dashed] (270:1.2) circle (0.125) node[anchor=west, xshift=4pt] {\eps{}-Neighborhood};
    \end{tikzpicture}
%}
\caption{A basic density-based clustering with \minneigh{} = 3 points and a unit \eps{} represented by the circles --- solid for the core neighborhoods and dashed for border.\protect\footnote{Recreated from \cite{johnson2015density}}
We see that the green points each contain 3 neighbors, but while the border points do not, they are within \eps{} of a core point and we thus cluster it along with the core points.
The (single) cluster that results from this set of datapoints are the green and gold points depicted.
}
\label{fig:density-based-clustering}
\end{figure}