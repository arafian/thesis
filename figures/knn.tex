%%%%%%%%%% MACROS %%%%%%%%%%
\newcommand{\coordsize}{10}
\newcommand{\gridextra}{0.5}
\newcommand{\circleradius}{0.4}
\newcommand{\unknowncolor}{white}
\newcommand{\acolor}{cpgreen}
\newcommand{\bcolor}{cpgold}
\newcommand{\ccolor}{cpgreengold}
\newcommand{\unknownchar}{?}
\newcommand{\achar}{$A$}
\newcommand{\bchar}{$B$}
\newcommand{\cchar}{$C$}
\newcommand{\ac}{\color{white}}
\newcommand{\drawunknownpoint}[1]{
\draw[thick, fill=\unknowncolor] (#1) circle (\circleradius) node[] {\unknownchar}
}
\newcommand{\drawapoint}[1]{
\draw[thick, fill=\acolor] (#1) circle (\circleradius) node[text=white] {\achar}
}
\newcommand{\drawbpoint}[1]{
\draw[thick, fill=\bcolor] (#1) circle (\circleradius) node[] {\bchar}
}
\newcommand{\drawcpoint}[1]{
\draw[thick, fill=\ccolor] (#1) circle (\circleradius) node[] {\cchar}
}

%%%%%%%%%% GRAPH %%%%%%%%%%
\begin{figure}
\centering
\begin{tikzpicture}[scale=1.25]
    \draw[step=1, gray, very thin] (0-\gridextra,0-\gridextra) grid (\coordsize+\gridextra,\coordsize+\gridextra);
    \draw[thick, ->] (0,0) -- (\coordsize+\gridextra,0);
    \draw[thick, ->] (0,0) -- (0,\coordsize+\gridextra);
    \foreach \x in {1,...,10} % edit here for the numbers
    \draw[shift={(\x,0)},color=black] (0pt,0pt) -- (0pt,-3pt) node[below, fill=white] {$\x$};
    \foreach \y in {1,...,10} % edit here for the numbers
    \draw[shift={(0,\y)},color=black] (0pt,0pt) -- (-3pt,0pt) node[left, fill=white] {$\y$};
    
    \drawunknownpoint{5,5};
    \drawbpoint{6,6}; % 1.414 B
    \drawapoint{3,6}; % 2.236 A
    \drawapoint{4,7}; % 2.236 A
    \drawcpoint{3,3}; % 3.000 C
    \drawbpoint{8,5}; % 2.828 B
    \drawbpoint{6,8}; % 3.162 B
    \drawcpoint{3,2}; % 3.606 C
    \drawcpoint{1,3}; % 4.472 C
    \drawcpoint{1,1}; % 5.657 C
    \drawapoint{1,10}; % 6.403 A
\end{tikzpicture}
\caption{Example graph of datapoints in a coordinate space. 
All but one of the datapoints have a class associated with them, \achar{}, \bchar{}, or \cchar{}. 
The class of one point, denoted by `\unknownchar{}', requires determination. 
\Compfuncs{} like \euclid{} or \manhattan{} may be the most appropriate way to compare these datapoints and \autoref{fig:knn} shows how \euclid{} and various \k{} can classify this point.}
\label{fig:knn-graph}
\end{figure}

%%%%%%%%%% TABLES %%%%%%%%%%
\begin{figure}
\centering
\subfloat[For \k{} = 4, the classification for the unknown-class datapoint is \achar{}.]{
\label{fig:knn:4}
\centering
\begin{tabular}{|c|c|c|c|}             \hline
    \k{} & Class   & Position & Similarity    \\ \hline
         & \unknownchar{}& (5,5) & 0.000       \\ \hline\hline
  \rowcolor{\bcolor}  1    & \bchar{}      & (6,6) & 1.414       \\ \hline
  \rowcolor{\acolor}  \ac2    & \ac\achar{}      & \ac(3,6) & \ac2.236       \\ \hline
  \rowcolor{\acolor}  \ac3    & \ac\achar{}      & \ac(4,7) & \ac2.236       \\ \hline
  \rowcolor{\ccolor}  4    & \cchar{}      & (3,3) & 2.828       \\ \hline
%  \rowcolor{\bcolor}  5    & \bchar{}      & (8,5) & 3.000       \\ \hline
%  \rowcolor{\bcolor}  6    & \bchar{}      & (6,8) & 3.162       \\ \hline
%  \rowcolor{\ccolor}  7    & \cchar{}      & (3,2) & 3.606       \\ \hline
%  \rowcolor{\ccolor}  8    & \cchar{}      & (1,3) & 4.472       \\ \hline
%  \rowcolor{\ccolor}  9    & \cchar{}      & (1,1) & 5.657       \\ \hline
%  \rowcolor{\acolor} \ac10    & \ac\achar{}      & \ac(1,10) & \ac6.403       \\ \hline
\end{tabular}
}
\\
\subfloat[For \k{} = 6, the classification for the unknown-class datapoint is \bchar{}.]{
\label{fig:knn:6}
\centering
\begin{tabular}{|c|c|c|c|}             \hline
    \k{} & Class   & Position & Similarity    \\ \hline
         & \unknownchar{}& (5,5) & 0.000       \\ \hline\hline
  \rowcolor{\bcolor}  1    & \bchar{}      & (6,6) & 1.414       \\ \hline
  \rowcolor{\acolor}  \ac2    & \ac\achar{}      & \ac(3,6) & \ac2.236       \\ \hline
  \rowcolor{\acolor}  \ac3    & \ac\achar{}      & \ac(4,7) & \ac2.236       \\ \hline
  \rowcolor{\ccolor}  4    & \cchar{}      & (3,3) & 2.828       \\ \hline
  \rowcolor{\bcolor}  5    & \bchar{}      & (8,5) & 3.000       \\ \hline
  \rowcolor{\bcolor}  6    & \bchar{}      & (6,8) & 3.162       \\ \hline
%  \rowcolor{\ccolor}  7    & \cchar{}      & (3,2) & 3.606       \\ \hline
%  \rowcolor{\ccolor}  8    & \cchar{}      & (1,3) & 4.472       \\ \hline
%  \rowcolor{\ccolor}  9    & \cchar{}      & (1,1) & 5.657       \\ \hline
%  \rowcolor{\acolor} \ac10    & \ac\achar{}      & \ac(1,10) & \ac6.403       \\ \hline
\end{tabular}
}
\\
\subfloat[For \k{} = 9, the classification for the unknown-class datapoint is \cchar{}.]{
\label{fig:knn:9}
\centering
\begin{tabular}{|c|c|c|c|}             \hline
    \k{} & Class   & Position & Similarity    \\ \hline
         & \unknownchar{}& (5,5) & 0.000       \\ \hline\hline
  \rowcolor{\bcolor}  1    & \bchar{}      & (6,6) & 1.414       \\ \hline
  \rowcolor{\acolor}  \ac2    & \ac\achar{}      & \ac(3,6) & \ac2.236       \\ \hline
  \rowcolor{\acolor}  \ac3    & \ac\achar{}      & \ac(4,7) & \ac2.236       \\ \hline
  \rowcolor{\ccolor}  4    & \cchar{}      & (3,3) & 2.828       \\ \hline
  \rowcolor{\bcolor}  5    & \bchar{}      & (8,5) & 3.000       \\ \hline
  \rowcolor{\bcolor}  6    & \bchar{}      & (6,8) & 3.162       \\ \hline
  \rowcolor{\ccolor}  7    & \cchar{}      & (3,2) & 3.606       \\ \hline
  \rowcolor{\ccolor}  8    & \cchar{}      & (1,3) & 4.472       \\ \hline
  \rowcolor{\ccolor}  9    & \cchar{}      & (1,1) & 5.657       \\ \hline
  %\rowcolor{\acolor} \ac10    & \ac\achar{}      & \ac(1,10) & \ac6.403       \\ \hline
\end{tabular}
}
\caption{A simple \kNNlong{} example, showing how the resultant classification of an unknown-class datapoint can change simply by adjusting the value of \k{}. This example uses the data in \autoref{fig:knn-graph} with similarity values calculated by the \euclid{} of the two points.}
\label{fig:knn}
\end{figure}